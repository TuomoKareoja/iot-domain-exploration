%%%%%%%%%%%%%%%%%%%%%%%%%%%%%%%%%%%%%%%%%
% Beamer Presentation
% LaTeX Template
% Version 1.0 (10/11/12)
%
% This template has been downloaded from:
% http://www.LaTeXTemplates.com
%
% License:
% CC BY-NC-SA 3.0 (http://creativecommons.org/licenses/by-nc-sa/3.0/)
%
%%%%%%%%%%%%%%%%%%%%%%%%%%%%%%%%%%%%%%%%%

%----------------------------------------------------------------------------------------
%	PACKAGES AND THEMES
%----------------------------------------------------------------------------------------

\documentclass[10pt]{beamer}

\mode<presentation> {

% The Beamer class comes with a number of default slide themes
% which change the colors and layouts of slides. Below this is a list
% of all the themes, uncomment each in turn to see what they look like.

\renewcommand{\familydefault}{\rmdefault}

\graphicspath{ {./figures/} }
\usepackage{graphicx} % Allows including images
\usepackage{booktabs} % Allows the use of \toprule, \midrule and \bottomrule in tables
\usepackage{tikz} % Allows the use of \toprule, \midrule and \bottomrule in tables

\usetheme{default}
% \usetheme{AnnArbor}
% \usetheme{Antibes}
% \usetheme{Bergen}
% \usetheme{Berkeley}
% \usetheme{Berlin}
% \usetheme{Boadilla}
% \usetheme{CambridgeUS}
% \usetheme{Copenhagen}
% \usetheme{Darmstadt}
% \usetheme{Dresden}
% \usetheme{Frankfurt}
% \usetheme{Goettingen}
% \usetheme{Hannover}
% \usetheme{Ilmenau}
% \usetheme{JuanLesPins}
% \usetheme{Luebeck}
% \usetheme{Madrid}
% \usetheme{Malmoe}
% \usetheme{Marburg}
% \usetheme{Montpellier}
% \usetheme{PaloAlto}
% \usetheme{Pittsburgh}
% \usetheme{Rochester}
% \usetheme{Singapore}
% \usetheme{Szeged}
% \usetheme{Warsaw}

% As well as themes, the Beamer class has a number of color themes
% for any slide theme. Uncomment each of these in turn to see how it
% changes the colors of your current slide theme.

% \usecolortheme{albatross}
\usecolortheme{beaver}
% \usecolortheme{beetle}
% \usecolortheme{crane}
% \usecolortheme{dolphin}
% \usecolortheme{dove}
% \usecolortheme{fly}
% \usecolortheme{lily}
% \usecolortheme{orchid}
% \usecolortheme{rose}
% \usecolortheme{seagull}
% \usecolortheme{seahorse}
% \usecolortheme{whale}
% \usecolortheme{wolverine}

%\setbeamertemplate{footline} % To remove the footer line in all slides uncomment this line
\setbeamertemplate{footline}[page number] % To replace the footer line in all slides with a simple slide count uncomment this line

\setbeamertemplate{navigation symbols}{} % To remove the navigation symbols from the bottom of all slides uncomment this line

% \setbeamertemplate{background}{
%     \tikz[overlay,remember picture]\node[opacity=0.4]at (current page.center){
%         \includegraphics[width=2cm]{iot_analytics_logo.jpg}
%         }}

\setbeamertemplate{background}{\tikz[overlay,remember picture]\node[anchor=south west,outer sep=0pt,inner sep=0pt] at (current page.south west) {\includegraphics[width=1.3cm]{iot_analytics_logo.jpg}};
}
}

%----------------------------------------------------------------------------------------
%	TITLE PAGE
%----------------------------------------------------------------------------------------

\title[Data Science in Blackwell]{Possibilities of Submetering Analysis} % The short title appears at the bottom of every slide, the full title is only on the title page

\author{Tuomo Kareoja} % Your name
\institute[IOT Analytics] % Your institution as it will appear on the bottom of every slide, may be shorthand to save space
{
IOT Analytics \\ % Your institution for the title page
\medskip
}
\date{\today} % Date, can be changed to a custom date

\begin{document}

\begin{frame}
\titlepage % Print the title page as the first slide
\end{frame}

\begin{frame}
\frametitle{Agenda} % Table of contents slide, comment this block out to remove it
\tableofcontents % Throughout your presentation, if you choose to use \section{} and \subsection{} commands, these will automatically be printed on this slide as an overview of your presentation
\end{frame}

%----------------------------------------------------------------------------------------
%	PRESENTATION SLIDES
%----------------------------------------------------------------------------------------

%------------------------------------------------
\section{Background}
%------------------------------------------------

%------------------------------------------------
\subsection{What is Submetering}
%------------------------------------------------

\begin{frame}
\frametitle{What is Submetering?}
% Uncomment the code on this slide to include your own image from the same directory as the template .TeX file.

\bigskip
{
    \centering
    \includegraphics[width=\textwidth,height=\textheight,keepaspectratio]{submetering.png}
    \par
}
\bigskip

\end{frame}

%------------------------------------------------
\subsection{What are the Possibilities of Submetering for IOT?}
%------------------------------------------------

\begin{frame}
\frametitle{What are the Possibilities of Submetering for IOT?}
% Uncomment the code on this slide to include your own image from the same directory as the template .TeX file.

Connected Submeters give a real time reading of the energy use in different parts of buildings
or even different device groups

\bigskip

This mean we can\ldots
\begin{enumerate}
    \item See what is actually spending that electricity
    \item Pinpoint possible failure points fast and precisely
    \item Gather time series data about the electricity use with a chosen granularity for further analysis
\end{enumerate}

\end{frame}


%------------------------------------------------
\section{Analysis}
%------------------------------------------------

%------------------------------------------------
\subsection{The Data}
%------------------------------------------------

\begin{frame}
\frametitle{The Data}

\begin{columns}[c] % The "c" option specifies centered vertical alignment while the "t" option is used for top vertical alignment

\column{.5\textwidth} % Left column and width

\begin{itemize}
    \item Minute level readings from 3 submeters and overall energy usage between December 2006 and November 2010 in House in Sceaux
    \item The three submeters cover the kitchen, the laundry room and the combination of the water-heater and air-conditioning
    \item Electricity use not covered by submeters can be calculated by deducting the submeter readings from overall electricity use
\end{itemize}

\column{.5\textwidth} % Right column and width

\bigskip
{
    \centering
    \includegraphics[width=\textwidth,height=\textheight,keepaspectratio]{sceaux.jpg}
    \par
}
\bigskip

\end{columns}

\end{frame}

%------------------------------------------------
\subsection{Missing Data and Other Problems}
%------------------------------------------------
\begin{frame}
\frametitle{Missing Data and Other Problems}

\begin{itemize}
    \item Grouping of submeters is based on room and not function
    \begin{itemize}
        \item Refrigerator is in the laundry room
        \item Water heater and air-conditioner not separated
    \end{itemize}
    \item 1.25 \% of values missing. Sometimes in stretches over a day
    \item Electricity use not covered by submeters makes up a over half of the energy consumption
\end{itemize}

% TODO add picture of data problem

\end{frame}

%------------------------------------------------
\subsection{Basic Stats}
%------------------------------------------------

\begin{frame}
\frametitle{Basic Stats}

Distribution by year month
Distribution by weekday
Distribution by hour

\end{frame}


%------------------------------------------------
\section{Possible Applications}
%------------------------------------------------

%------------------------------------------------
\subsection{Predictive Analytics}
%------------------------------------------------

\begin{frame}
\frametitle{Predictive Analytics}

Problematic when taking too detailed view
Work well on a larger scale

\end{frame}

%------------------------------------------------
\subsection{Dashboards}
%------------------------------------------------

\begin{frame}
\frametitle{Dashboards}

Customer facing application for monitoring electricity use
can be combined with predictive analytics

\end{frame}


%------------------------------------------------
\section{Recommendations for Next Steps}
%------------------------------------------------


\begin{frame}
\frametitle{Recommendations for Next Steps}

Make sure that the submeter groupings are decided with the customer
Build customer facing dashboard with some predictive analytics features

\end{frame}

\begin{frame}
\frametitle{The End}

\LARGE{\centerline{Questions?}}

\end{frame}

%----------------------------------------------------------------------------------------

\end{document}
