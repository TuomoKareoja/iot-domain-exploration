%%%%%%%%%%%%%%%%%%%%%%%%%%%%%%%%%%%%%%%%%
% Beamer Presentation
% LaTeX Template
% Version 1.0 (10/11/12)
%
% This template has been downloaded from:
% http://www.LaTeXTemplates.com
%
% License:
% CC BY-NC-SA 3.0 (http://creativecommons.org/licenses/by-nc-sa/3.0/)
%
%%%%%%%%%%%%%%%%%%%%%%%%%%%%%%%%%%%%%%%%%

%----------------------------------------------------------------------------------------
%	PACKAGES AND THEMES
%----------------------------------------------------------------------------------------

\documentclass[10pt]{beamer}

\mode<presentation> {

% The Beamer class comes with a number of default slide themes
% which change the colors and layouts of slides. Below this is a list
% of all the themes, uncomment each in turn to see what they look like.

\renewcommand{\familydefault}{\rmdefault}

\graphicspath{ {./figures/} }
\usepackage{graphicx} % Allows including images
\usepackage{booktabs} % Allows the use of \toprule, \midrule and \bottomrule in tables
\usepackage{tikz} % Allows the use of \toprule, \midrule and \bottomrule in tables
\usepackage{hyperref}
\usepackage[printwatermark]{xwatermark}

\usetheme{default}
% \usetheme{AnnArbor}
% \usetheme{Antibes}
% \usetheme{Bergen}
% \usetheme{Berkeley}
% \usetheme{Berlin}
% \usetheme{Boadilla}
% \usetheme{CambridgeUS}
% \usetheme{Copenhagen}
% \usetheme{Darmstadt}
% \usetheme{Dresden}
% \usetheme{Frankfurt}
% \usetheme{Goettingen}
% \usetheme{Hannover}
% \usetheme{Ilmenau}
% \usetheme{JuanLesPins}
% \usetheme{Luebeck}
% \usetheme{Madrid}
% \usetheme{Malmoe}
% \usetheme{Marburg}
% \usetheme{Montpellier}
% \usetheme{PaloAlto}
% \usetheme{Pittsburgh}
% \usetheme{Rochester}
% \usetheme{Singapore}
% \usetheme{Szeged}
% \usetheme{Warsaw}

% As well as themes, the Beamer class has a number of color themes
% for any slide theme. Uncomment each of these in turn to see how it
% changes the colors of your current slide theme.

% \usecolortheme{albatross}
\usecolortheme{beaver}
% \usecolortheme{beetle}
% \usecolortheme{crane}
% \usecolortheme{dolphin}
% \usecolortheme{dove}
% \usecolortheme{fly}
% \usecolortheme{lily}
% \usecolortheme{orchid}
% \usecolortheme{rose}
% \usecolortheme{seagull}
% \usecolortheme{seahorse}
% \usecolortheme{whale}
% \usecolortheme{wolverine}

%\setbeamertemplate{footline} % To remove the footer line in all slides uncomment this line
\setbeamertemplate{footline}[page number] % To replace the footer line in all slides with a simple slide count uncomment this line

\setbeamertemplate{navigation symbols}{} % To remove the navigation symbols from the bottom of all slides uncomment this line

% \setbeamertemplate{background}{
%     \tikz[overlay,remember picture]\node[opacity=0.4]at (current page.center){
%         \includegraphics[width=2cm]{iot_analytics_logo.jpg}
%         }}

\setbeamertemplate{background}{\tikz[overlay,remember picture]\node[anchor=south west,outer sep=0pt,inner sep=0pt] at (current page.south west) {\includegraphics[width=1.2cm]{iot_analytics_logo.jpg}};
}

}

%----------------------------------------------------------------------------------------
%	TITLE PAGE
%----------------------------------------------------------------------------------------

\title[Data Science in Blackwell]{What does Submetering Data Show?} % The short title appears at the bottom of every slide, the full title is only on the title page

\author{Tuomo Kareoja} % Your name
\institute[IOT Analytics] % Your institution as it will appear on the bottom of every slide, may be shorthand to save space
{
IOT Analytics \\ % Your institution for the title page
\medskip
}
\date{\today} % Date, can be changed to a custom date

\begin{document}

\begin{frame}
\titlepage % Print the title page as the first slide
\end{frame}

\begin{frame}
\frametitle{Agenda} % Table of contents slide, comment this block out to remove it
\tableofcontents % Throughout your presentation, if you choose to use \section{} and \subsection{} commands, these will automatically be printed on this slide as an overview of your presentation
\end{frame}

%----------------------------------------------------------------------------------------
%	PRESENTATION SLIDES
%----------------------------------------------------------------------------------------

%------------------------------------------------
\section{Trends and Patterns}
%------------------------------------------------

%------------------------------------------------
\subsection{Overall Trends?}
%------------------------------------------------

\begin{frame}
\frametitle{Overall Trends}

{
    \centering
    \includegraphics[width=\textwidth,height=\textheight,keepaspectratio]{energy_use_overall_trends.png}
    \par
}

\end{frame}

%------------------------------------------------
\subsection{Monthly Patterns}
%------------------------------------------------

\begin{frame}
\frametitle{Monthly Patterns}

\bigskip
{
    \centering
    \includegraphics[width=\textwidth,height=\textheight,keepaspectratio]{monthly_seasonality.png}
    \par
}
\bigskip

\end{frame}

%------------------------------------------------
\subsection{Weekly Patterns}
%------------------------------------------------

\begin{frame}
\frametitle{Weekly Patterns}

\bigskip
{
    \centering
    \includegraphics[width=\textwidth,height=\textheight,keepaspectratio]{weekly_seasonality.png}
    \par
}
\bigskip

\end{frame}

%------------------------------------------------
\subsection{Daily Patterns}
%------------------------------------------------

\begin{frame}
\frametitle{Daily Patterns}

\bigskip
{
    \centering
    \includegraphics[width=\textwidth,height=\textheight,keepaspectratio]{daily_seasonality.png}
    \par
}
\bigskip

\end{frame}


%------------------------------------------------
\section{Peak into Individual Days}
%------------------------------------------------

%------------------------------------------------
\subsection{A Typical Summer Day}
%------------------------------------------------

\begin{frame}
\frametitle{A Typical Summer Day}

\bigskip
{
    \centering
    \includegraphics[width=\textwidth,height=\textheight,keepaspectratio]{20080607.png}
    \par
}
\bigskip

\end{frame}

%------------------------------------------------
\subsection{Christmas Holiday}
%------------------------------------------------

\begin{frame}
\frametitle{Christmas Holiday}

\bigskip
{
    \centering
    \includegraphics[width=\textwidth,height=\textheight,keepaspectratio]{20091224.png}
    \par
}
\bigskip

\end{frame}

%------------------------------------------------
\section{Predictive Modelling}
%------------------------------------------------

%------------------------------------------------
\subsection{Predictive Analytics}
%------------------------------------------------

\begin{frame}
\frametitle{Hour Level Model - Predicting Hours for One Month}

\bigskip
{
    \centering
    \includegraphics[width=\textwidth,height=\textheight,keepaspectratio]{model_comparison_timeseries_hourmodel_hours.png}
    \par
}
\bigskip

\bigskip
{
    \centering
    \includegraphics[width=\textwidth,height=\textheight,keepaspectratio]{model_comparison_table_hourmodel_hours.png}
    \par
}
\bigskip

\end{frame}

\begin{frame}
\frametitle{Hour Level Model - Predicting Days for One Month}

\bigskip
{
    \centering
    \includegraphics[width=\textwidth,height=\textheight,keepaspectratio]{model_comparison_timeseries_hourmodel_days.png}
    \par
}
\bigskip

\bigskip
{
    \centering
    \includegraphics[width=\textwidth,height=\textheight,keepaspectratio]{model_comparison_table_hourmodel_days.png}
    \par
}
\bigskip



\end{frame}

\begin{frame}
\frametitle{Hour Level Model - Predicting for One Month}

\bigskip
{
    \centering
    \includegraphics[width=\textwidth,height=\textheight,keepaspectratio]{model_comparison_timeseries_hourmodel_months.png}
    \par
}
\bigskip

\bigskip
{
    \centering
    \includegraphics[width=\textwidth,height=\textheight,keepaspectratio]{model_comparison_table_hourmodel_months.png}
    \par
}
\bigskip


\end{frame}

\begin{frame}
\frametitle{Day Level Model - Predicting Days for One Month}

\bigskip
{
    \centering
    \includegraphics[width=\textwidth,height=\textheight,keepaspectratio]{model_comparison_timeseries_daymodel_days.png}
    \par
}
\bigskip

\bigskip
{
    \centering
    \includegraphics[width=\textwidth,height=\textheight,keepaspectratio]{model_comparison_table_daymodel_days.png}
    \par
}
\bigskip


\end{frame}

\begin{frame}
\frametitle{Day Level Model - Predicting for One Month}

\bigskip
{
    \centering
    \includegraphics[width=\textwidth,height=\textheight,keepaspectratio]{model_comparison_timeseries_daymodel_months.png}
    \par
}
\bigskip

\bigskip
{
    \centering
    \includegraphics[width=\textwidth,height=\textheight,keepaspectratio]{model_comparison_table_daymodel_months.png}
    \par
}
\bigskip


\end{frame}

%------------------------------------------------
\section{Conclusions}
%------------------------------------------------


\begin{frame}
\frametitle{Conclusions}

\begin{enumerate}
    \item Privacy!
    \begin{itemize}
        \item Individual days show an intimate picture of how people live their lives
        \item Even aggregated measures show private information
    \end{itemize}
    \item Security! \(\Rightarrow \) A burglar could see when somebody is at home
    \item Categorizing patterns with a machine model would probably be possible, but would require
    lots of manual effort, because we would have to label the observations ourselves
    \item
\end{enumerate}

\end{frame}

\begin{frame}
\frametitle{The End}

\LARGE{\centerline{Questions?}}

\end{frame}

%----------------------------------------------------------------------------------------

\end{document}
